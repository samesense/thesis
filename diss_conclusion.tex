\chapter{Reflections and perspectives} \label{chapter:conclusion}

In this dissertation, we presented three projects that introduced new
observations about the nature of virus-host networks and generated
testable hypothesis for further virus-host network discoveries. In the
first project, we showed that host pathways targeted by HIV could be
predicted using peptide motifs on HIV sequences. Our HIV-human
interaction models have predicted new pathways and interactions that
may be important for HIV infection. In our second project, we examined
the docking between HIV proteins and human mitogen-activated protein
kinases, which may be important for HIV replication, and proposed
docking sites on HIV substrates that can be evaluated in the lab. In
our third project, we contributed to the network biology field by
addressing the observation that viruses target host network hub
proteins, and asking if viruses had a preferential interaction with
intermodular or intramodular hubs. By demonstrating a preference of
intramodular hubs over intermodular hubs for HIV and influenza virus,
we aided the systems study of biological networks by providing more
evidence for the distinction between intermodular and intramodular
hubs, which is currently under debate
\cite{taylor09,batada06,batada07,agarwal09,han04}. Furthermore,
the virus intramodular hub preference promotes the study of which
intramodular hub properties are important for viral infection. In this
final chapter, we review the work presented in this dissertation and
address how it can be used as a basis for future investigations of
virus-host networks.

% the third result seems a bit weak

\section{Review of our work}

\subsubsection{Prediction of HIV virus-host protein interactions using virus and host sequence motifs}

In Chapter \ref{chapter:predict} we presented a peptide motif based
virus-host interaction prediction method, and tested its ability to
accurately recover HIV-human interactions listed in the NCBI HIV-Human
Protein Interaction Database \cite{fu09,ptak08}. We motivated our work
by outlining the importance of predicted virus-host interactions for
guiding experimental studies of virus-host interactions
\cite{skrabanek2008computational,jansen2003bayesian,lee2004probabilistic}. The
virus-host networks that emerge from virus-host interaction
experiments have been used to annotate virus proteins of unknown
function and compare different viral strategies for dealing with the
host immune system \cite{navratil-system,calderwood07}. We showed that
our predicted interactions had significant overlap with interactions in
the NCBI database, and that virus targeted proteins from our
predictions overlapped significantly with a set of host proteins that
are important for HIV replication \cite{bushman09}. Our HIV-human
interaction predictions were further validated in that the human
proteins in the interactions occupied many of the same biological
pathways as the human proteins shown to be targeted by HIV in the
validated NCBI interactions. We showed that our predicted virus
targeted proteins were also enriched in some pathways not known to interact
with HIV, providing new potential directions for the study of
virus-host networks. Our work for this chapter has been further
summarized in a review by Chan et al.\ \cite{chan09}.

Our prediction work has generated hypotheses about new virus targeted
host pathways and provided a list of host proteins whose interactions
with virus proteins may be essential for HIV replication. The cell
cycle, Jak-STAT, cytoskeletal regulation, and tight junction KEGG
pathways were all significantly enriched in our predictions for some
HIV protein, but the corresponding enrichment was not significant for
the validated virus-host interactions from NCBI. These pathways offer
new hypotheses for cell processes that HIV might need to
target. Combining the predicted virus-host interactions with the
results form siRNA screens searching for host factors that are
important for HIV replication also leads to new hypotheses. Predicted
virus targeted proteins that are implicated in an siRNA screen can be
tested to see if preventing the interaction has an affect on HIV
replication.  Furthermore, we have identified several protein binding
sites on host and virus proteins that may be guiding HIV-human
interactions that are essential for replication. Following up on the
proposed virus-host interactions important for replication and the
suggested virus targeted pathways will help construct a more complete
HIV-human interaction network that can facilitate future HIV studies.

%% This chapter has also raised questions about the nature
%% of HIV-human interactions. While evaluating our predictions, we
%% observed that including our predicted first level interactors as well
%% as our predictions for second level interactors performed better than
%% using the first level interactors alone. Our validation set includes
%% both protein binding and regulatory information. Perhaps the second
%% level interactors are being regulated by HIV proteins, while the first
%% level interactors are binding with HIV proteins.

\subsubsection{A bioinformatics approach reveals possible MAPK docking motifs on HIV proteins}

In Chapter \ref{chapter:mapk} we continued our work with the
hypothesis that virus proteins use host peptide motifs to interact
with host proteins, and focused on the peptide motif that acts as a
substrate docking site for mitogen-activated protein kinases (MAPKs)
ERK1 and ERK2 \cite{bardwell09}. Our work with MAPK and virus proteins
was motivated by the importance of MAPK phosphorylation of HIV
substrates MA and Vif in infection \cite{yang99,bukrinskaya96}. We
observed that HIV proteins MA, Rev, Tat, and Vif, while documented to
be phosphorylated by ERK1 and ERK2, were missing the accepted MAPK
docking motif. The HIV Nef protein had hits for the MAPK docking motif
pattern, but further investigation of Nef's structure suggested that
these sites were not functional. We revealed that modifications of the
accepted MAPK docking motif pattern would yield peptide motif patterns
that annotated all HIV proteins phosphorylated by ERK1 and ERK2. As an
argument that our proposed docking motifs were functional, we showed
that they were enriched on human MAPK ERK1 and ERK2 substrates, and we
demonstrated that \textit{in silico} docking of MAPK ERK1 and HIV MA via the
proposed docking site produced a protein complex that aligned the
active site of ERK1 with a possible phosphorylation site on MA. The
locations of our proposed docking motifs on HIV proteins serve as
testable sites that mediate MAPK and HIV protein interactions. If
functional, our proposed docking sites will aid in the search for
small-molecule drugs that prevent HIV protein phosphorylation by MAPKs
ERK1 and ERK2.

The work in this chapter is important for future work with virus-host
interactions. First, it serves as outline of computational steps for
the modification and verification of host peptide motif for use on
virus proteins. Second, this chapter motivates more questions
concerning the presence of host peptide motifs on virus proteins. Are
there other peptide motifs, like the MAPK docking site, that are
variations of documented host peptide motifs? How is the virus
utilization of variant host peptide motif patterns beneficial to the
virus? Answering these questions will yield more insights into the
nature of virus-host networks.

\subsubsection{Modularity in protein interaction network hubs predicts viral host-pathogen interactions}

In Chapter \ref{chapter:hubs} we bridged the gap between studies of
hubs in single organism networks and work done with virus-host
networks to determine if host hub modularity, i.e., the presence of
two hub types in networks, played a role in virus-host
interactions. We reaffirmed the debated existence of intermodular and
intramodular hubs in single organism networks using three human
protein-protein interaction networks, and confirmed some of the
properties that have been observed for intermodular and intramodular
hubs, such as the preference of intramodular hubs to be parts of
proteins complexes \cite{fraser05}. We showed that the despite being
debated \cite{taylor09,batada06,batada07,agarwal09,han04}, the
inter/intramodular hub distinction is important for network systems
biology by demonstrating that HIV and influenza virus proteins have a
significant interaction preference for intramodular hubs. We ended
this chapter by proposing that the virus intramodular hub preference
is caused by a virus preference to interact with hubs that are part of
host protein complexes.

The virus intramodular hub preference described in this work promotes
questions about the features of host proteins that are targeted by
viruses. Intramodular hubs evolve at slower rates
\cite{fraser2005modularity}, and are more structured than intermodular
hubs
\cite{ekman2006properties,singh2007role,tokuriki2009viral}. Perhaps
viruses are targeting these hub features in addition to the
intramodular hub complex feature. More work should be done to
investigate the importance of these intramodular hub features in
guiding virus protein preference.

\section{Future work}
The work outlined here will help with future studies of virus-host
networks. In addition to the experimental work suggested above, our
work motivates other computational studies. Here we introduce four
promising computational extensions to this work.

\subsubsection{Predicting virus-host integrations using peptide motifs and additional information}

One of the draw backs of predicting virus-host interactions using
peptide motifs is the high number of false positive predictions. This
problem can be alleviated by including additional information. Some of
this information is provided virus protein structures. HIV protein
structures have already been used to predict virus-human interactions
\cite{doolittle2010structural}. The peptide motif method can be
supplemented by this work where virus protein structures are
available. The peptide motif method can also easily fit into a
prediction method that utilizes virus-host network motifs
\cite{hivNetMotifs}. Network motifs are over-represented patterns of
interaction involving two or more proteins. Network motifs have been
successfully used to predict yeast protein interactions
\cite{albert2004conserved}. It is likely that combining a similar
approach using virus-host network motifs with the peptide motif based
interaction prediction method will be able to predict virus-host
interactions with fewer false positives.

\subsubsection{Database of HIV mutations and their effects on virus-host interactions}

Another fault of the peptide motif based virus-host interaction
prediction method is the lack of evidence that an individual peptide
sequence is responsible for a virus-host protein interaction. In the
first aim of the dissertation, we showed that the conservation of
peptide motifs on the HIV Nef protein sequences was not due to chance,
and used this significant conservation to argue that conserved peptide
motifs on HIV proteins were mediating virus-host interactions. In the
second aim of the dissertation, we proposed that the statistical
enrichment of a peptide motif on the interaction neighbors of a
protein was evidence that the peptide motif was being used in
interactions with some neighbor proteins.

The task of finding functional peptide motifs would be simplified if
there was a database of mutations on virus proteins and their effects
on virus-host interactions, identifying the functional motif hits
would be much easier. The NCBI HIV-Human Interaction Database has some
of the information needed to make such a database, but it is poorly
organized. A database of HIV protein mutations can be built by taking
all the source articles from the NCBI database, mining them for
paragraphs mentioning mutations and virus-host interactions, and then
having a large community of biologists annotation of these paragraphs
with information describing the mutation and its effect on virus-host
interactions. Using these virus mutations to find functional peptide
motifs, and using functional peptide motifs with structure and network
guides would make a better model of virus-host interactions.

\subsubsection{Investigating the role of virus proteins as intermodular hubs in host protein interaction networks}

Highly connected proteins in the human protein interaction network
come in two types, intermodular hubs that modulate the activities of
human complexes, and intramodular hubs that play important roles in
these complexes \cite{han04,taylor09}. When combined with other
studies of virus proteins, the work in this dissertation motivates the
hypothesis that virus proteins act as intermodular hub proteins in
virus-host networks. There is much evidence to support this
claim. First, a study of influenza-human interactions revealed that
influenza proteins had more interactions with host proteins than
expected by chance, indicating that virus proteins might be hub
proteins in the virus-host interaction network
\cite{shapira2009physical}. Second, in Chapter \ref{chapter:hubs} we
demonstrated that HIV and influenza proteins prefer to interact with
intramodular hubs. We suggested that this hub preference was actually
a preference to interact with host protein complexes, which is the
same preference seen for intermodular hubs. Third, intermodular hubs
are highly unstructured, or disordered, proteins compared to
intramodular hubs \cite{ekman2006properties,singh2007role}. It has
been observed that 25 RNA virus proteins with structures available in
the Protein Data Bank \cite{berman02} had large regions of protein
disorder \cite{tokuriki2009viral}. Specific cases of virus protein
disorder involved the HIV Rev and matrix proteins
\cite{turner1999structural,goh2008protein}. The basic protein segment
of HIV Rev that binds HIV RNA is unstructured when alone in solution,
but adopts an $\alpha$-helix when bound to RNA
\cite{turner1999structural}. The HIV matrix protein is highly
disordered, and this might help the virus in evading the immune system
\cite{goh2008protein}. Fourth, intermodular hubs have more peptide
motifs than intramodular hubs \cite{taylor09}. In Chapter
\ref{chapter:predict}, we showed that not only do virus proteins have
many conserved host peptide motifs, but these motifs are possibly
functional because they can be used to predict interactions with virus
proteins.

Testing the hypothesis that virus proteins act as intermodular hubs in
the virus-host network is important because it has implications for
antiviral therapies. Disordered proteins must be tightly regulated in
host cells because their binding promiscuity makes then highly
sensitive to changes in their concentration
\cite{vavouri2009intrinsic,gsponer2008tight}. There is evidence that
concentration changes also affect HIV infection. In an HIV infected
cell, HIV proteins Env, Nef, and Vpu regulate the presence of the HIV
CD4 receptor at the cell's membrane
\cite{levesque2004role}. Downregulating the HIV CD4 receptor at the
cell's membrane is thought to decrease the chance of reinfection by
more HIV virions, which would stress the host cell's pathways without
resulting in the production of more virus particles
\cite{arold2001dynamic,harris1999hiv}. CD4 downregulation by HIV hints
that the tight regulation of HIV protein expression is necessary for
successful replication. More extensive investigation is needed to
validate this hypothesis and develop methods to alter virus protein
regulation for therapy purposes.

\subsubsection{Examining the effects of host environment on influenza virus peptide motif usage}

Influenza virus has the ability to infect a number of host organisms,
including human, chicken, swine, and horse. Nucleotide sequences of
the 2009 H1N1 pandemic influenza have been shown to have a substitution
bias that depends on the host organism in which it resides
\cite{solovyov2010host}. Based on our work in Chapters
\ref{chapter:predict} and \ref{chapter:mapk}, we propose a project to
examine the possibility of an influenza virus peptide motif usage bias
that correlates with host organism. Our results in Chapter
\ref{chapter:predict} suggested that peptide motifs on virus proteins
are important for guiding virus-host protein interactions. Case
studies focusing on single peptide motifs on certain virus proteins
have shown that some virus proteins use host peptide motifs to
interact with host proteins to accomplish necessary steps in the viral
life-cycle \cite{kadaveru08}. Regions of some influenza proteins have
already been found to evolve differently in different hosts. The
identities of four amino acids in an influenza polymerase component
have been found to differ consistently between mammalian and avian
hosts, and it has been suggested that these differences affect
interactions with host proteins \cite{yamada2010biological}. In light
of the importance of peptide motifs for virus-host interactions, we
suggest conducting a study of differential peptide motif usage among
influenza viruses infection mammalian and avian hosts. Such a study
will give insights into the selective pressures on influenza proteins,
which can aid in drug design and the determination of which organism
an influenza strain has originated.

%% Here we demonstrate that the usage of peptide motifs differs between
%% influenza groups and host organisms. We show that an influenza group's
%% peptide motif profile if closest that of its host.

%% Peptide sequence usage has been used to profile prokaryotic proteomes,
%% and the distances between profiles accuratly recovers the known
%% prokaryotic phylogeny \cite{jun2010whole}.

%% Host disordered proteins have more peptide motifs than other
%% proteins \cite{gsponer2009rules}.

%% In addition to better prediction methods, more having more virus-host
%% interaction data will allow different 

%% Our work on HIV and peptide motifs has been reviewed
%% \cite{chan09}.

%% Next steps will include the investigation of the role of host
%% microRNAs (miRNAs) in the virus host network, where host miRNAs can
%% modulate the virus gene expression \cite{ghosh2009cellular}.

%% Our work can also be extended to study hosts other than human. A
%% forward-genetic screen has identified 57 E. coli genes needed for
%% lambda phage replication \cite{maynard2010forward}. The study of
%% bacteria-virus interactions is useful because of the symbiotic
%% relationship that exists in human intestines, which may determine
%% human diet \cite{citeulike:7479351}.
