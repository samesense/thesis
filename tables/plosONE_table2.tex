\begin{table}\footnotesize
\begin{center}
\begin{tabular}{|l|c|c|c|c||l|c|c|c|c|}
\hline
\multicolumn{5}{|c||}{A (Da)} & \multicolumn{5}{|c|}{B (Db)}\\
\hline
VP & A1 & B & C & Total & VP & A1 & B & C & Total\\
\hline
MA&	6&	14&	50&	34&	MA&	1&	1&	1&	1\\
Nef&	97&	89&	98&	93&	Nef&	96&	89&	91&	90\\
Rev&	1&	9&	1&	3&	Rev&	0&	0&	1&	0\\
Tat&	0&	0&	0&	0&	Tat&	67&	91&	44&	59\\
Vif&	92&	6&	97&	50&	Vif&	1&	5&	61&	27\\
\hline
\hline
\multicolumn{5}{|c||}{C (Dc)} & \multicolumn{5}{|c|}{D (Dd)}\\
\hline
VP & A1 & B & C & Total & VP & A1 & B & C & Total\\
MA&	95&	96&	96&	96&	MA&	95&	99&	97&	97\\
Nef&	100&	100&	100&	100&	Nef&	60&	92&	86&	87\\
Rev&	100&	99&	96&	97&	Rev&	100&	100&	100&	100\\
Tat&	69&	93&	45&	61&	Tat&	100&	99&	100&	99\\
Vif&	100&	100&	100&	100&	Vif&	92&	87&	100&	92\\
\hline
  \end{tabular}
\end{center}
\caption[MAPK docking pattern hits on HIV proteins]{\small We searched sequences of HIV proteins using the four MAPK docking site patterns in Table \ref{tbl:plosONE1:patterns}. Here we present the percentages of HIV subtype sequences with these docking site patterns. The Da and Db  patterns were found on the majority of Nef sequences, but they were missing from some subtypes of the other HIV proteins. The Dc pattern occurred on the majority of MA, Nef, Rev, and Vif subtypes. The Dd  motif had hits on most sequences of all HIV proteins. \label{tbl:plosONE2:percents}}
\end{table}

