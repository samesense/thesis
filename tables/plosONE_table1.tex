\begin{table}\footnotesize
\begin{center}
  \begin{tabular}{|l|c|c|c|c|}
  \hline
  Motif	& Pattern & Phos(\%) & ERK(\%) & p-val\\
  \hline
Da &	[KR]\{0,2\}[KR].\{0,2\}[KR].\{2,4\}[ILVM].[ILVF] & 558 (43) & 56 (45) & 0.299\\
Db&	[KR]\{2,3\}.\{1,6\}[ILVM].[ILVF]&	513 (39)&	51 (41)&	0.348\\
Da U Db&	-&	620 (48)&	69 (56)&	0.030\\
Dc&	[KR].\{2,6\}[ILVM].[ILVF]&	841 (65)&	92 (75)&	0.007\\
Dd&	[KR].\{1,3\}[KR]\{2\}&	694 (53)&	70 (57)&	0.231\\
\hline
  \end{tabular}
\end{center}
\caption[MAPK docking pattern hits on human proteins]{\small Using
  each of the MAPK docking site patterns, we scanned phosphorylated
  substrates in the Database of Post Translational Modifications
  (dbPTM) \cite{lee06}. We show the number of phosphorylated
  substrates with pattern matches (Phos column) as well as results for
  ERK1/2 substrates (ERK column). We used Fisher's exact test to
  calculate a p-value for the enrichment of pattern hits on ERK1/2
  substrates compared to all other phosphorylated proteins. The
  standard docking site patterns, Da and Db, were not enriched on
  ERK1/2 substrates, but the union of these patterns, Da U Db, was
  enriched. Dc, but not Dd, was found to be enriched on ERK1/2
  substrates. \label{tbl:plosONE1:patterns}}
\end{table}

