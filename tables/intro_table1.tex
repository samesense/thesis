\begin{table}\footnotesize
\begin{center}
  \begin{tabular}{|l|c|c|}
  \hline
  Virus & siRNA screen hits & Protein interactions (Interacting human proteins) \\
  \hline
HIV & 850 & 2652 (887) \\
HCV & 318 & 477 (414) \\
Influenza & 295 & 339 (230) \\
Papillomavirus & NA & 229 (94) \\
EBV & NA & 173 \\
KSHV & NA & 173 \\
VZV & NA & 123 \\
\hline
  \end{tabular}
\end{center}
\caption[Experimentally determined virus-host interactions]{\small For
  each virus, we show the number interactions that occur between virus
  and human proteins, and count the number of host factors necessary
  for infection (siRNA screen hits). Of the viruses in the table, HCV,
  HIV, and influenza virus have the most experimentally determined
  virus-host interactions. siRNA screen hits describe interactions
  between a virus and a human gene. siRNA screens have only been
  conducted for HCV, HIV, and the influenza virus. Protein
  interactions cover only direct interactions, like protein binding or
  protein modifications, such as the phosphorylation of a virus
  protein by a host kinase. A single human protein can be involved in
  multiple interactions with different virus proteins, so the total
  number of unique human proteins involved in virus-human interactions
  is given in parentheses. While only direct interactions are listed
  here, HIV has a total of 3950 direct and indirect (e.g. regulatory,
  induced protein modification) interactions with 1439 human proteins
  \cite{fu09}. \label{tbl:intro:counts}}
\end{table}
