  \subsection{Human interactome}
    We converted the human I2D (accessed October 2009), human STRING
    (version 8.2), and HPRD (release 8) interactomes to networks of
    Entrez Gene IDs using ID mapping provided by UniProt
    \cite{citeulike:6008733}, gProfiler \cite{reimand2007g}, and
    NCBI's eutils. Network hubs were found for each interactome
    separately by locating the lowest degree in the top 20\% of
    connected genes, and taking all genes with at least this
    degree. Bimodal curves for each interactome's hub average PCC
    distribution were found using gradient descent to minimize the
    log-likelihood of a binned distribution. The number of bins for
    each distribution was determined by dividing the total number of
    data points by a bin divisor. For individual networks, we used a
    bin divisor of 11 for I2D and 10 for STRING and HPRD. For
    distributions made using hubs present in at least two networks, we
    fit a bimodal curve using a bin divisor of 6 for STRING and 9 for
    I2D and HPRD.

  \subsection{Virus-host interactomes}
    We gathered HIV-human protein interactions from NCBI and
    VirusMINT. Both virus-host interaction datasets label interactions
    by type. We focused on interaction types describing direct protein
    interactions, or modifications. We used VirusMINT interactions
    labeled with MINT interaction IDs 0006 (anti bait
    coimmunoprecipitation), 0007 (anti tag coimmunoprecipitation),
    0018 (two hybrid), 0019 (coimmunoprecipitation), 0027
    (cosedimentation), 0045 (experimental interaction detection), 0096
    (pull down), 0416 (fluorescence microscopy), 0424 (protein kinase
    assay), 0435 (protease assay), 0515 (methyltransferase assay),
    0889 (acetylation assay). For the NCBI database, we used
    acetylated by, acetylates, binds, cleaved by, cleaves, degraded
    by, degrades, dephosphorylates, methylated by, myristoylated by,
    phosphorylated by, phosphorylates, ubiquitinated by edges. We
    gathered HIV siRNA results from two sources covering four studies,
    and converted these hits to Entrez Gene IDs. Three studies were
    summarized in Table 4 of the supplementary document supplied by
    Bushman et al.\ at http://www.hostpathogen.org
    \cite{bushman09}. The fourth study was taken from Figure S2 in an
    article by Yeung et al. \cite{yeung09}.

    HCV-human protein interactions were taken from de Chassey et
    al. \cite{dechassey08}. HCV siRNA results were taken as genes that
    showed a decrease in infection, listed in supplementary material
    SD1 and SD2 columns B-E \cite{Li09}.

    Influenza-human protein interactions and siRNA hits were taken
    from a study of host factors involved in influenza replication
    \cite{konig2009human}. For the protein interactions, we ignored
    interactions where no virus protein was specified.

   \subsection{Peptide motif and SMART domain annotations}
    Protein sequences form three human interactomes, I2D, STRING, and
    HPRD, were scanned for peptide motifs and SMART domains. Protein
    sequences for the STRING and HPRD interactomes were taken from
    their respective databases. Protein sequences for the I2D network
    were taken from UniProt. Human proteins were scanned for SMART
    domains using batch access. Human proteins were annotated with the
    136 peptide motifs described in the ELM Resource by downloading
    regular expressions for each motif from the resource, and matching
    them against all human protein sequences. The networks used in
    this study were composed of Entrez Gene IDs, and multiple proteins
    may correspond to one Entrez Gene ID. For peptide motif and SMART
    domain annotations for each Entrez Gene ID, we averaged the
    annotations of all the proteins for which it coded.

  %% \subsection{HCV-human interaction predictions}
  %%   Multiple alignments for HCV proteins F, CORE, E1, E2, NS2, NS3,
  %%   NS4A, NS4B, NS5A, NS5B, and P7 were obtained from the Los Alamos
  %%   National Laboratory HCV sequence database
  %%   (http://hcv.lanl.gov/content/sequence/NEWALIGN/align.html). HCV
  %%   proteins were annotated with 136 ELMs using regular
  %%   expressions. For each of the 53 ELMs found to be conserved on at
  %%   least 90\% of an HCV protein's alignment, we found either a
  %%   PROSITE domain \cite{hulo08} known to interact with the ELM, or a
  %%   set of proteins known to interact with the ELM \cite{evans09}. We
  %%   used the ELM Resource and ScanProsite to annotate proteins from
  %%   I2D, STRING, and HPRD. Predictions were made using the method
  %%   described by Evans et al. \cite{evans09}.

%ScanProsite \cite{decastro06}
