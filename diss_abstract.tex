Virus-host interactions are being cataloged at an increasing rate
using protein interaction assays and small interfering RNA screens for
host factors necessary for infection. These interactions can be viewed
as a network, where genes or proteins are nodes, and edges correspond
to associations between them. Virus-host interaction networks will
eventually support the study and treatment of infection, but first
require more data and better analysis techniques. This dissertation
targets these goals with three aims. The first aim tackles the lack of
data by providing a method for the computational prediction of
virus-host protein interactions. We show that HIV-human protein
interactions can be predicted using documented human peptide motifs
found to be conserved on HIV proteins from different subtypes. We find
that human proteins predicted to bind to HIV proteins are enriched in
both documented HIV targeted proteins and pathways known to be
utilized by HIV. The second aim seeks to improve peptide motif
annotation on virus proteins, starting with the docking site for
protein kinases ERK1 and ERK2, which phosphorylate HIV proteins during
infection. We find that the docking site motif, in spite of being
suggestive of phosphorylation, is not present on all HIV subtypes for
some HIV proteins, and we provide evidence that two variations of the
docking site motif could explain phosphorylation. In the third aim, we
analyze virus-host networks and build on the observation that viruses
target host hub proteins. We show that of the two hub types, date and
party, HIV and influenza virus proteins prefer to interact with the
latter. The methods presented here for prediction and motif
refinement, as well as the analysis of virus targeted hubs, provide a
useful set of tools and hypotheses for the study of virus-host
interactions.
